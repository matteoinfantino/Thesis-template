
%**************************************************************
% Acronimi
%**************************************************************
\renewcommand{\acronymname}{Acronimi e abbreviazioni}

\newacronym[description={\glslink{apig}{Application Program Interface}}]
    {api}{API}{Application Program Interface}

\newacronym[description={\glslink{umlg}{Unified Modeling Language}}]
    {uml}{UML}{Unified Modeling Language}

%**************************************************************
% Glossario
%**************************************************************
%\renewcommand{\glossaryname}{Glossario}

\newglossaryentry{turn over}
{
    name=\glslink{turn over}{Turn over aziendale},
    text=Turn over,
    sort=turn over,
    description={Tasso di ricambio del personale, ovvero il flusso di persone in ingresso e in uscita da un'azienda}
}

\newglossaryentry{big data}
{
    name=\glslink{big data}{Big data},
    text=big data,
    sort=Big Data,
    description={Tecniche e metodologie di analisi di grandi quantità di dati ovvero la capacità di estrapolare, analizzare e mettere in relazione un’enorme mole di dati eterogenei, strutturati e non strutturati, per scoprire i legami tra fenomeni diversi e prevedere quelli futuri. Si parla di big data quando si ha un insieme talmente grande e complesso di dati che richiede la definizione di nuovi strumenti e metodologie per estrapolare, gestire e processare informazioni entro un tempo ragionevole}
}

\newglossaryentry{cloud computing}
{
    name=\glslink{cloud computing}{Cloud computing},
    text=cloud computing,
    sort=Cloud Computing,
    description={Con il termine Cloud Computing (ing. “nuvola informatica”) si indica un
paradigma di erogazione di servizi offerti on demand da un fornitore ad un cliente finale
attraverso la rete Internet}
}

\newglossaryentry{internet of things}
{
    name=\glslink{internet of things}{Internet of things},
    text=internet of things,
    sort=Internet of Things,
    description={Nelle telecomunicazioni, Internet of things (internet degli oggetti) è
il modo di riferirsi all’estensione di Internet al mondo degli oggetti e dei luoghi concreti.
Al giorno d’oggi infatti sempre più oggetti comuni hanno la possibilità di interfacciarsi
con il web, a partire da lavatrici, frigoriferi per arrivare a tutta la casa controllabile e
accessibile da remoto}
}

\newglossaryentry{information and communication technology}
{
    name=\glslink{information and communication technology}{Information and Communication Technology},
    text=information and communication technology,
    sort=Information and Communication Technology,
    description={Con ICT si indica il settore legato allo sviluppo delle strutture internet e
mobile, in particolare per ciò che riguarda gli aspetti di progettazione e realizzazione delle
componenti fisiche o di quelle digitali. Più in generale, le ICT comprendono l’insieme
dei metodi e delle tecniche utilizzate nella trasmissione, ricezione ed elaborazione di dati
e informazioni digitali, ampiamente diffusesi a partire dalla cosiddetta Terza rivoluzione
industriale: hardware,software e tecnologie ICT costituiscono oggi le tre componenti
principali delsettore IT}
}

\newglossaryentry{contact tracing}
{
    name=\glslink{contact tracing}{Contact tracing},
    text=contact tracing,
    sort=contact tracing,
    description={Il contact tracing è il tracciamento dei contatti nell'ambito della sanità pubblica, tramite il quale si identificano le persone che potrebbero essere venute a contatto con una persona infetta. Nell'ambito informatico con applicazione per il contact tracing si intende un programma tipicamente per smartphone che raccolga le informazioni relative ai contatti tra utenti}
}

\newglossaryentry{bluetooth LE}
{
    name=\glslink{bluetooth LE}{Bluetooth LE},
    text=bluetooth LE,
    sort=bluetooth LE,
    description={Il bluetooth LE è una tecnologia wireless con consumi energetici notevolmente ridotti rispetto al bluetooth classico, seppur con intervalli di comunicazione simili. Le sue caratteristiche si adattano a molteplici applicazioni che richiedono elevati utilizzi nel tempo del bluetooth, come per il contact tracing}
}

\newglossaryentry{UML}
{
    name=\glslink{UML}{UML},
    text=UML,
    sort=uml,
    description={in ingegneria del software \emph{UML, Unified Modeling Language} (ing. linguaggio di modellazione unificato) è un linguaggio di modellazione e specifica basato sul paradigma object-oriented. L'\emph{UML} svolge un'importantissima funzione di ``lingua franca'' nella comunità della progettazione e programmazione a oggetti. Gran parte della letteratura di settore usa tale linguaggio per descrivere soluzioni analitiche e progettuali in modo sintetico e comprensibile a un vasto pubblico}
}

\newglossaryentry{distributed ledger}
{
    name=\glslink{distributed ledger}{Distributed ledger},
    text=distributed ledger,
    sort=Distributed ledger,
    description={Il distributed ledger è un database condiviso e sincronizzato attraverso svariati nodi. Ogni nodo è autorizzato ad aggiornare il ledger in modo indipendente ma sotto il controllo consensuale degli altri. Non vi è un'autorità centrale come nei database tradizionali, ma l'incorrutibilità è garantita da un algoritmo di consenso}
}

\newglossaryentry{algoritmo di consenso}
{
    name=\glslink{algoritmo di consenso}{Algoritmo di consenso},
    text=algoritmo di consenso,
    sort=algoritmo di consenso,
    description={Un algoritmo di consenso è un meccanismo attraverso il quale i nodi assicurano la validazione di un blocco nella rete. È fondamentale nelle blockchain perché la mancanza di un ente che convalidi le transazioni deve coincidere con la sicurezza che le transazioni avvengano correttamente e le regole del protocollo siano seguite. Gli algorimi di consenso più conosciuti sono il Proof of Work e il Proof of Stake}
}

\newglossaryentry{single point of failure}
{
    name=\glslink{single point of failure}{Single point of failure},
    text=single point of failure,
    sort=single point of failure,
    description={Un single point of failure è un singolo punto del sistema il cui malfunzionanto può causare problemi all'intera operatività}
}

\newglossaryentry{permissionless}
{
    name=\glslink{permissionless}{Permissionless},
    text=permissionless,
    sort=Permissionless,
    description={Una blockchain che viene così definita non richiede
alcuna autorizzazione per poter accedere alla rete, eseguire delle transazioni o partecipare
alla verifica e creazione di un nuovo blocco}
}

\newglossaryentry{permissioned}
{
    name=\glslink{permissioned}{Permissioned},
    text=permissioned,
    sort=permissioned,
    description={Una blockchain che viene così definita permette solo a utenti autenticati e autorizzati di accedere alla rete, eseguire delle transazioni o partecipare alla verifica e creazione di un nuovo blocco}
}

\newglossaryentry{miner}
{
    name=\glslink{miner}{Miner},
    text=miner,
    sort=miner,
    description={In una blockchain con algoritmo di consenso Proof of Work, si definisce miner un nodo che risolve i problemi computazionali richiesti per la validazione di un blocco, dietro un compenso ricevuto a lavoro ultimato}
}

\newglossaryentry{Bitcoin}
{
    name=\glslink{Bitcoin}{Bitcoin},
    text=Bitcoin,
    sort=Bitcoin,
    description={Bitocoin è una blockchain creata nel 2009 da Satoshi Nakamoto che utilizza l'omonima criptovaluta. Utilizza un ledger distribuito fra i nodi per tenere traccia delle transazioni, validate tramite algoritmo Proof of Work}
}

\newglossaryentry{Ethereum}
{
    name=\glslink{Ethereum}{Ethereum},
    text=Ethereum,
    sort=Ethereum,
    description={Piattaforma decentralizzata ideata da Vitalik Buterik nel 2015, famosa per la possibilità di eseguire smart contracts in blockchain}
}

\newglossaryentry{proof of work}
{
    name=\glslink{proof of work}{Proof of work},
    text=proof of work,
    sort=proof of work,
    description={Algoritmo di consenso utilizzato da diverse criptovalute, come Bitcoin,
Ethereum e Litecoin per raggiungere un accordo decentralizzato tra diversi nodi nel
processo di aggiunta di un blocco specifico alla blockchain. Tale algoritmo obbliga i miners
a risolvere dei problemi matematici estremamente complessi e computazionalmente
difficili per poter aggiungere blocchi alla blockchain}
}

\newglossaryentry{double spending}
{
    name=\glslink{double spending}{Double spending},
    text=double spending,
    sort=double spending,
    description={Il double spending è una potenziale criticità delle criptovalute per cui lo stesso token digitale viene speso più di una volta}
}

\newglossaryentry{proof of stake}
{
    name=\glslink{proof of stake}{Proof of stake},
    text=proof of stake,
    sort=proof of stake,
    description={Algoritmo di consenso basato sul possesso di criptovalute: gli account con un gran numero di token hanno più possibilità di generare un blocco valido. È meno utilizzato rispetto al Proof of Work, ma è in grande sviluppo per via di numerosi vantaggi rispetto agli altri algoritmi di consenso, come la scalabilità e il minor costo e consumo energetico}
}

\newglossaryentry{token}
{
    name=\glslink{token}{Token},
    text=token,
    sort=token,
    description={In blockchain un token è un gettone virtuale emesso da un'organizzazione che rappresenta un'unità di valore}
}

\newglossaryentry{turing completi}
{
    name=\glslink{turing completi}{Linguaggio Turing completo},
    text=Turing completi,
    sort=turing-completi,
    description={Un linguaggio è detto Turing completo quando è in grado di eseguire qualunque programma che una macchina di Turing può esegiore, dato sufficiente tempo e memoria}
}

\newglossaryentry{applicazioni decentralizzate}
{
    name=\glslink{applicazioni decentralizzate}{Applicazioni decentralizzate},
    text=applicazioni decentralizzate,
    sort=applicazioni decentralizzate,
    description={Un'applicazione decentralizzata è un programma eseguito su un sistema distribuito come le blockchain}
}

\newglossaryentry{criptovaluta}
{
    name=\glslink{criptovaluta}{Criptovaluta},
    text=criptovaluta,
    sort=criptovaluta,
    description={La criptovaluta è una rappresentazione digitale di valore basata sulla crittografia}
}

\newglossaryentry{fee}
{
    name=\glslink{fee}{Fee},
    text=fee,
    sort=fee,
    description={In ambito blockchain una fee è un pagamento richiesto per le operazioni effettuato che ha lo scopo di dare una ricompensa al miner che valida il blocco, in modo da sostenere il sistema}
}

\newglossaryentry{hash}
{
    name=\glslink{hash}{Hash},
    text=hash,
    sort=hash,
    description={L'hash è una funzione non invertibile che mappa una stringa di lunghezza arbitraria in una stringa di lunghezza fissa. Viene utilizzzata in ambito blockchain per l'algoritmo Proof of Work: calcolare un'hash con delle caratteristiche definite a priori è un lavoro computazionale oneroso e viene utilizzato per validare i blocchi come prova del lavoro effettuato}
}


\newglossaryentry{best practices}
{
    name=\glslink{best practices}{Best practices},
    text=best practices,
    sort=best practices,
    description={Prassi che secondo l'esperienza personale o di studi abbia dimostrato di garantire il miglior risultato in determinate circostanze}
}

\newglossaryentry{Ethereum Virtual Machine}
{
    name=\glslink{Ethereum Virtual Machine}{Ethereum Virtual Machine},
    text=Ethereum Virtual Machine,
    sort=Ethereum Virtual Machine,
    description={L'Ethereum Virtual Machine è la macchine virtuale per lo sviluppo e la gestione di smart contracts in Ethereum. Opera in modo separato dalla rete, ossia il codice gestito dalla Virtual Machine non ha accesso alla rete}
}

\newglossaryentry{indici di contatto}
{
    name=\glslink{indici di contatto}{Indici di contatto},
    text=indici di contatto,
    sort=indici di contatto,
    description={Nell'applicazione SyncTrace l'indice di contatto è un numero calcolato in funzione di distanza e tempo di contatto tra due persone, utilizzato per registrare il livello di contatto e il relativo rischio}
}

\newglossaryentry{gas}
{
    name=\glslink{gas}{Gas},
    text=gas,
    sort=gas,
    description={Unità di misura utilizzata in Ethereum per misurare la complessità delle operazioni e ricevere un'adeguata fee}
}

\newglossaryentry{gas price}
{
    name=\glslink{gas price}{Gas price},
    text=gas price,
    sort=gas price,
    description={Il gas price è un valore in criptovaluta assegnato alla singola unità di gas; viene utilizzato per dare un valore monetario a un'operazione, in modo da pagare i miners}
}

\newglossaryentry{gas limit}
{
    name=\glslink{gas limit}{Gas limit},
    text=gas limit,
    sort=gas limit,
    description={Il gas limit è una soglia di sicurezza utilizzata per prevenire loop infiniti in blockchain; se un'operazione supera il gas limit impostato, la transazione fallisce, ma il lavoro effettuato dai miners viene comunque pagato}
}

\newglossaryentry{repository}
{
    name=\glslink{repository}{Repository},
    text=repository,
    sort=repository,
    description={Ambiente di sistema informativo in cui vengono conservati e gestiti file, documenti e metadati relativi ad un'attività di progetto}
}

\newglossaryentry{deployment}
{
    name=\glslink{deployment}{Deploy},
    text=deployment,
    sort=deploy,
    description={Termine con il quale si indica il caricamento di un contratto all'interno della rete Ethereum}
}

\newglossaryentry{Proof of concept}
{
    name=\glslink{Proof of concept}{Proof of concept},
    text=Proof of concept,
    sort=Proof of concept,
    description={Per Proof of Concept, abbreviato spesso in PoC, si intende una realizzazione incompleta o abbozzata di un determinato progetto o metodo, allo
scopo di provarne la fattibilità o dimostrare la fondatezza di alcuni principi o concetti costituenti}
}

\newglossaryentry{gwei}
{
    name=\glslink{gwei}{Wei},
    text=gwei,
    sort=Wei,
    description={La più piccola unità in cui è possibile suddividere un Ether. Possiede numerosi multipli, come il gwei, molto utilizzato per comodità di espressione nel prezzo del gas}
}

\newglossaryentry{Ether}
{
    name=\glslink{Ether}{Ether},
    text=Ether,
    sort=ether,
    description={Criptovaluta utilizzata all'interno della blockchain Ethereum}
}

\newglossaryentry{Metamask}
{
    name=\glslink{Metamask}{Metamask},
    text=Metamask,
    sort=Metamask,
    description={Metamask è un estensione browser che permette di interagire con la blockchain Ethereum. Mette a disposizione un wallet per depositare e inviare Ether nella rete}
}

\newglossaryentry{Trello}
{
    name=\glslink{Trello}{Trello},
    text=Trello,
    sort=Trello,
    description={Software gestionale in stile kanban che consente di lavorare in modo più organizzato e collaborativo}
}

