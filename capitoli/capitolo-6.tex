% !TEX encoding = UTF-8
% !TEX TS-program = pdflatex
% !TEX root = ../tesi.tex

%**************************************************************
\chapter{Conclusioni}
\label{cap:conclusioni}

%**************************************************************
\section{Raggiungimento degli obiettivi}
Nella sezione \hyperref[sec:obiettivi]{2.4.4} sono stati presentati gli obiettivi proposti dal tutor aziendale a inizio stage.
Di seguito vengono riportate le tabelle degli obiettivi con il loro stato di completamento
\begin{center}
	\begin{longtable}{| c | p{23em} | c | }
		\caption{Tabella degli obiettivi obbligatori}
		\label{tab:obiettivi-obbligatori}\\
		\hline
		\textbf{Obiettivo} & \centering\textbf{Descrizione} & Stato\\
		\endfirsthead
		\hline
		\textbf{Obiettivo} & \centering\textbf{Descrizione} & Stato\\
		\endhead
		\endfoot
		
		\hline
		O01    & Acquisizione competenze sulle tecnologie blockchain, in particolare Ethereum  & Completato\\
		\hline
		O02    & Capacità di progettazione e analisi di smart contract & Completato\\
		\hline
		O03    & Capacità di raggiungere gli obiettivi richiesti in autonomia, seguendo il programma preventivato) & Completato\\
		\hline
		O04    & Portare a termine l'implementazione di almeno l'80\% degli sviluppi previsti & Completato\\
		\hline
	\end{longtable}
\end{center}

\begin{center}
	\begin{longtable}{| c | p{23em} | c |}
		\caption{Tabella degli obiettivi desiderabili}
		\label{tab:obiettivi-desiderabili}\\
		\hline
		\textbf{Obiettivo} & \centering\textbf{Descrizione} & Stato\\
		\endfirsthead
		\hline
		\textbf{Obiettivo} & \centering\textbf{Descrizione} & Stato\\
		\endhead
		\endfoot
		
		\hline
		D01    & Portare a termine il lavoro di studio della portabilità di Ethereum su dispositivi mobili & Completato\\
		\hline
		D02    & Portare a termine l’implementazione completa degli sviluppi previsti & Completato\\
		\hline
	\end{longtable}
\end{center}
	
\begin{center}
	\begin{longtable}{| c | p{23em} | c |}
		\caption{Tabella degli obiettivi facoltativi}
		\label{tab:obiettivi-facoltativi}\\
		\hline
		\textbf{Obiettivo} & \centering\textbf{Descrizione} & Stato\\
		\endfirsthead
		\hline
		\textbf{Obiettivo} & \centering\textbf{Descrizione} & Stato\\
		\endhead
		\endfoot
		
		\hline
		F01    & Completare l'installazione di un peer su un dispositivo mobile Android & Completato\\
		\hline
	\end{longtable}
\end{center}

%**************************************************************
\section{Conoscenze acquisite}
Le conoscenze acquisite durante lo stage sono molteplici, sia a livello teorico che pratico. 
\paragraph{Blockchain}\mbox{}\\
La prima parte del percorso è stata incentrata sullo studio della tecnologia \textit{blockchain}, in tutte le sue sfaccettature. Essendo un campo mai affrontato durante il percorso universitario e non avendo particolari conoscenze pregresse, ho dedicato particolare attenzione allo studio delle \textit{blockchain}, sia per interesse personale, che per comprendere al meglio i concetti generali su cui lo stage era focalizzato. 
\paragraph{Ethereum e Solidity}\mbox{}\\
La fase successiva ha riguardato lo studio della piattaforma \textit{Ethereum} e di tutti gli strumenti utilizzati per lo sviluppo di \textit{smart contract} per la \textit{blockchain}. Grazie allo studio teorico delle \textit{blockchain}, è stato facile comprendere il funzionamento della famosa piattaforma. Il linguaggio di programmazione utilizzato per gli \textit{smart contract} è stato \textit{Solidity}. A prima vista è risultato facile da comprendere, ma padroneggiare fino in fondo la programmazione nel contesto \textit{blockchain} ha richiesto tempo, soprattutto per quanto riguarda l'ottimizzazione del codice.
\paragraph{Utilizzo smart contract in applicazione mobile e web}\mbox{}\\
Dopo aver sviluppato lo \textit{smart contract} per il \textit{contact tracing}, mi è stato richiesto di integrarlo nelle applicazioni SyncTrace. Farlo è stato molto utile e formativo perché non mi sono limitato a sviluppare uno \textit{smart contract}, ma ho anche imparato a sviluppare un'applicazione decentralizzata sfruttando la \textit{blockchain}. Inoltre l'integrazione ha richiesto la collaborazione con il resto del team. 

%**************************************************************
\section{Valutazione personale}
Il bilancio dello stage effettuato presso Sync Lab è senza dubbio positivo. Gli argomenti trattati mi hanno interessato sin dalla scelta del percorso e non hanno deluso le mie aspettative. La \textit{blockchain} è una tecnologia che si sta enormemente sviluppando negli ultimi anni e questa esperienza è stata un'opportunità perfetta per approfondirla. Sono riuscito a comprendere i pregi e i difetti dello sviluppare un'applicazione decentralizzata al giorno d'oggi, quando conviene e quando no, al di là del caso d'uso affrontato nel progetto. Sicuramente è una tecnologia in continua evoluzione e i prossimi anni possono rappresentare una svolta per questo campo. Sarà molto interessante osservare come il passaggio di \textit{Ethereum} al \textit{Proof of Stake} possa aumentare le opportunità e le potenzialità della \textit{blockchain}.\\ 
Valuto positivamente l'esperienza anche per il rapporto con il mondo del lavoro. Lo stage rappresenta un'occasione per affacciarsi al contesto aziendale, completamente diverso da quello universitario. Purtroppo la pandemia ha costretto l'azienda a svolgere la maggior parte delle attività da remoto e questo ha certamente limitato il valore positivo che il tirocinio può portare. Ciononostante, ho apprezzato l'organizzazione aziendale anche in questo periodo, in particolar modo quando è stato possibile svolgere intere giornate lavorative in presenza.
 


