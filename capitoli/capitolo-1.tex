% !TEX encoding = UTF-8
% !TEX TS-program = pdflatex
% !TEX root = ../tesi.tex

%**************************************************************
\chapter{Introduzione}
\label{cap:introduzione}
%**************************************************************

Introduzione al contesto applicativo.\\

\noindent Esempio di utilizzo di un termine nel glossario \\
\gls{api}. \\

\noindent Esempio di citazione in linea \\
\cite{site:agile-manifesto}. \\

\noindent Esempio di citazione nel pie' di pagina \\
citazione\footcite{womak:lean-thinking} \\

%**************************************************************
\section{L'azienda}

Sync Lab S.R.L. è una società fondata nel 2002 a Napoli come software house e diventata rapidamente un'azienda di consulenza nel dominio dell'Information and Communication Technology(ICT).
Oggi Sync Lab ha raggiunto un’ampia diffusione sul territorio attraverso le sue cinque sedi: Napoli, Roma, Milano, Padova e Verona.
L’organico aziendale è andato aumentando in modo continuo e rapido, in relazione all’apertura delle varie sedi ed alla progressiva crescita delle stesse, raggiungendo le cento collaborazioni nel 2007 e superando le duecento nel 2016. Con l’aiuto dei suoi specialisti che lavorano continuamente in maniera sincronizzata, collaborativa e disciplinata, Sync Lab propone ai suoi clienti un'ampia gamma di prodotti nei settori mobile, videosorveglianza e sicurezza delle infrastrutture informatiche aziendali.\\
Le politiche di assunzione hanno reso Sync Lab un punto di riferimento per coloro che intendono avviare o far evolvere in chiave professionale la loro carriera: l’alto tasso di assunzione post-stage ed il basso turn-over testimoniano la voglia di condividere il progetto comune, assumendo ruoli e responsabilità che possono essere offerti solo da un processo evolutivo così intenso.

\begin{figure}[h]
\caption{Logo Sync Lab}
\centering
\includegraphics[width=0.8\textwidth]{./immagini/logo-synclab.jpg}
\end{figure}

\subsection{Servizi offerti}
La principale attività di Sync Lab è la consulenza tecnologica, un processo continuo di identificazione e messa in opera di soluzioni su misura, finalizzate alla creazione di valore. I principali servizi che fornisce l'azienda sono:
\begin{itemize}
	\item{Business Consultancy;}
	\item{Project Financing;}
	\item{IT Consultancy.}
\end{itemize}
L’offerta di consulenza specialistica trova le punte di eccellenza nella progettazione di
architetture software avanzate, siano esse per applicativi di dominio, per sistemi di Supporto
al Business (BSS), per sistemi di integrazione (EAI/SOA) o per sistemi di monitoraggio applicativo/territoriale. Il laboratorio Ricerca e Sviluppo (RD) dell’azienda è sempre al passo con
i nuovi paradigmi tecnologici e di comunicazione, ad esempio Big Data, Cloud Computing,
Internet of Things (IoT), Mobile e Sicurezza IT, per supportare i propri clienti nella creazione
ed integrazione di applicazioni, processi e dispositivi. Le attività in ambito Educational ed
RD hanno permesso di acquisire una profonda conoscenza degli strumenti di finanza agevolata
fruendone direttamente ed interagendo con enti di supporto ai progetti innovativi dei propri
clienti.
L’azienda, grazie alla rete di relazioni a livello nazionale ed internazionale, ha ottenuto
importanti finanziamenti in progetti RD europei (FP7 e H2020).

\subsection{Prodotti offerti}
Dalla sua creazione fin ad oggi, Sync Lab ha sviluppato diversi prodotti
garantendone sempre la qualità grazie alle certicazioni ISO 9001, ISO 14001, ISO 27001, OHSAS 18001.\\\\
I prodotti offerti sono i seguenti:
\begin{itemize}
	\item{\textbf{SynClinic: }rappresenta il sistema informativo sanitario per la gestione integrata di tutti i processi clinici e amministrativi ospedalieri, cliniche e case di cura. Utilizzabile sia in sia on premises, gestisce, organizza e monitora tutte le fasi del percorso di cura del paziente, diventando supporto indispensabile per il personale
nella gestione del rischio clinico;}
	\item{\textbf{DPS 4.0: }rappresenta una soluzione web per una compliance continua che fa uso della GDPRg Privacy acronimo inglese General Data Protection Regulation che è un regolamento attraverso il quale la Commissione Europea intende rafforzare la protezione dei dati personali di cittadini dell’Unione Europea ;}
	\item{\textbf{StreamLog: }rappresenta un sistema finalizzato al soddisfacimento dei requisiti fissati dal Garante, ovvero è in grado di effettuare il controllo degli accessi degli utenti ai sistemi in modo semplice ed efficace. Il sistema è basato su framework open source allo stato dell’arte e, in particolare, su un’innovativa tecnologia di
"streaming", frutto del laboratorio di Ricerca e Sviluppo Sync Lab;}
	\item{\textbf{StreamCrusher: }tecnologia che aiuta ad essere ben informati su quando bisogna prendere decisioni di business, a identificare velocemente criticità e a riorganizzare
i processi in base a nuove esigenze;}
	\item{\textbf{Wave: }nato dal laboratorio di Ricerca e Sviluppo Sync Lab, si propone come integrazione sinergica tra i mondi della Videosorveglianza e quello dei Sistemi Informativi}
\end{itemize} L’approfondita conoscenza di processi e tecnologie, maturata in esperienze altamente
significative e qualificanti, fornisce l’expertise e il know-how necessari per gestire progetti di
elevata complessità, dominando l’intero ciclo di vita:
\begin{itemize}
	\item{Studio di fattibilità;}
	\item{Progettazione;}
	\item{Implementazione;}
	\item{Governance;}
	\item{Post Delivery.}
\end{itemize}

\subsection{Settori di impiego}
Sync Lab si sta sempre più specializzando in vari settori d’impiego: dal mondo banking
all’assurance con una nicchia importante nell’ambito sanità in cui vanta un prodotto d’eccellenza
per la gestione delle cliniche private. L’azienda inoltre ha recentemente fondato un
reparto collegato Sync Security che si occupa del mondo della cyber security e
sicurezza informatica in genere.

%**************************************************************
\section{Organizzazione del testo}

\begin{description}
    \item[{\hyperref[cap:processi-metodologie]{Il secondo capitolo}}] descrive ...
    
    \item[{\hyperref[cap:descrizione-stage]{Il terzo capitolo}}] approfondisce ...
    
    \item[{\hyperref[cap:analisi-requisiti]{Il quarto capitolo}}] approfondisce ...
    
    \item[{\hyperref[cap:progettazione-codifica]{Il quinto capitolo}}] approfondisce ...
    
    \item[{\hyperref[cap:verifica-validazione]{Il sesto capitolo}}] approfondisce ...
    
    \item[{\hyperref[cap:conclusioni]{Nel settimo capitolo}}] descrive ...
\end{description}

Riguardo la stesura del testo, relativamente al documento sono state adottate le seguenti convenzioni tipografiche:
\begin{itemize}
	\item gli acronimi, le abbreviazioni e i termini ambigui o di uso non comune menzionati vengono definiti nel glossario, situato alla fine del presente documento;
	\item per la prima occorrenza dei termini riportati nel glossario viene utilizzata la seguente nomenclatura: \emph{parola}\glsfirstoccur;
	\item i termini in lingua straniera o facenti parti del gergo tecnico sono evidenziati con il carattere \emph{corsivo}.
\end{itemize}