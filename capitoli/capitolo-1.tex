% !TEX encoding = UTF-8
% !TEX TS-program = pdflatex
% !TEX root = ../tesi.tex

%**************************************************************
\chapter{Introduzione}
\label{cap:introduzione}
%**************************************************************

%Introduzione al contesto applicativo.\\

%\noindent Esempio di utilizzo di un termine nel glossario \\
%\gls{api}. \\

%\noindent Esempio di citazione in linea \\
%\cite{site:agile-manifesto}. \\

%\noindent Esempio di citazione nel pie' di pagina \\
%citazione\footcite{womak:lean-thinking} \\

%**************************************************************
\section{L'azienda}

Sync Lab S.R.L. è una società fondata nel 2002 a Napoli come \textit{software house} e diventata rapidamente un'azienda di consulenza nel dominio dell'\emph{\gls{information and communication technology}}\glsfirstoccur (ICT).
Oggi Sync Lab ha raggiunto un’ampia diffusione sul territorio attraverso le sue cinque sedi: Napoli, Roma, Milano, Padova e Verona.
L’organico aziendale è andato aumentando in modo continuo e rapido, in relazione all’apertura delle varie sedi e alla progressiva crescita delle stesse, raggiungendo le cento collaborazioni nel 2007 e superando le duecento nel 2016. Con l’aiuto dei suoi specialisti, che lavorano continuamente in maniera sincronizzata, collaborativa e disciplinata, Sync Lab propone ai clienti un'ampia gamma di prodotti nei settori mobile, videosorveglianza e sicurezza delle infrastrutture informatiche aziendali.\\
Le politiche di assunzione hanno reso Sync Lab un punto di riferimento per coloro che intendono avviare o far evolvere in chiave professionale la loro carriera: l'azienda, infatti, vanta un alto tasso di assunzione post-stage e un basso \emph{\gls{turn over}}\glsfirstoccur.\\\\

\begin{figure}[h]
\caption{Logo Sync Lab}
\centering
\includegraphics[width=0.8\textwidth]{./immagini/logo-synclab.jpg}
\end{figure}

\subsection{Servizi offerti}
La principale attività di Sync Lab è la consulenza tecnologica, un processo continuo di identificazione e messa in opera di soluzioni su misura, finalizzate alla creazione di valore. I principali servizi che fornisce l'azienda sono:
\begin{itemize}
	\item{\textit{Business Consultancy};}
	\item{\textit{Project Financing};}
	\item{\textit{IT Consultancy}.}
\end{itemize}
L’offerta di consulenza specialistica trova le punte di eccellenza nella progettazione di
architetture software avanzate, siano esse per applicativi di dominio, per sistemi di supporto
al business, per sistemi di integrazione o per sistemi di monitoraggio. Il laboratorio ricerca e sviluppo dell’azienda è sempre al passo con
i nuovi paradigmi tecnologici e di comunicazione, come \emph{\gls{big data}}\glsfirstoccur, \emph{\gls{cloud computing}}\glsfirstoccur,
\emph{\gls{internet of things}}\glsfirstoccur (IoT), al fine di supportare i propri clienti nella creazione
e integrazione di applicazioni, processi e dispositivi. 
L’azienda, grazie alla rete di relazioni a livello nazionale ed internazionale, ha ottenuto
importanti finanziamenti in progetti europei (FP7 e H2020).\\
L’approfondita conoscenza di processi e tecnologie, maturata in esperienze altamente
significative e qualificanti, permette all'azienda di gestire progetti di
elevata complessità, dominando l’intero ciclo di vita del software:
\begin{itemize}
	\item{Studio di fattibilità;}
	\item{Analisi dei requisiti;}
	\item{Progettazione;}
	\item{Implementazione;}
	\item{Manutenzione.}
\end{itemize}

\subsection{Prodotti offerti}
Dalla sua creazione fin ad oggi, Sync Lab ha sviluppato diversi prodotti,
garantendone sempre la qualità grazie alle certicazioni ISO 9001, ISO 14001, ISO 27001, OHSAS 18001.\\\\
I principali prodotti offerti sono i seguenti:
\begin{itemize}
	\item{\textbf{SynClinic: }è un sistema informativo sanitario per la gestione integrata di tutti i processi clinici e amministrativi di ospedali, cliniche e case di cura. Gestisce, organizza e monitora tutte le fasi del percorso di cura del paziente, diventando supporto indispensabile per il personale sanitario;}
	\item{\textbf{DPS 4.0: }rappresenta una soluzione web per una compliance continua, gestendo la \textit{GDPR Privacy} (\textit{General Data Protection Regulation}), regolamento attraverso il quale la Commissione Europea intende rafforzare la protezione dei dati personali dei cittadini dell’Unione Europea;}
	\item{\textbf{StreamLog: }è un sistema finalizzato al soddisfacimento dei requisiti fissati dal garante. È in grado di effettuare il controllo degli accessi degli utenti ai sistemi in modo semplice ed efficace. Il sistema è basato su \textit{framework open source} e su un’innovativa tecnologia di streaming, frutto del laboratorio di ricerca e sviluppo Sync Lab;}
	\item{\textbf{StreamCrusher: }tecnologia che aiuta ad essere ben informati su quando bisogna prendere decisioni di business, a identificare velocemente criticità e a riorganizzare i processi in base a nuove esigenze;}
	\item{\textbf{Wave: } è un software che si propone come integrazione sinergica tra i mondi della videosorveglianza e quello dei sistemi informativi territoriali, abilitando il controllo totale dell'area da sorvegliare.}
\end{itemize}

\subsection{Settori di impiego}
Sync Lab è specializzata in vari settori d’impiego: dal mondo \textit{banking}, all’\textit{assurance}, con una nicchia importante nell’ambito sanità, in cui vanta un prodotto d’eccellenza per la gestione delle cliniche private. L’azienda, inoltre, ha recentemente fondato un reparto collegato Sync Security che si occupa del mondo della sicurezza informatica.

%**************************************************************
\section{Organizzazione del testo}

\begin{description}
    \item[{\hyperref[cap:descrizione-stage]{Il secondo capitolo}}] fornisce una panoramica del percorso di stage intrapreso, focalizzandosi poi sulle tecnologie coinvolte nel progetto.
    
    \item[{\hyperref[cap:analisi-requisiti]{Il terzo capitolo}}] illustra i requisiti emersi in fase di analisi.
    
    \item[{\hyperref[cap:progettazione-codifica]{Il quarto capitolo}}] approfondisce la progettazione e l'implementazione del contratto.
    
    \item[{\hyperref[cap:integrazione-synctrace]{Il quinto capitolo}}] approfondisce l'integrazione del contratto nel software di contact tracing sviluppato. dall'azienda
    
    \item[{\hyperref[cap:conclusioni]{Nel sesto capitolo}}] si analizza il percorso effettuato e si stipulano le conclusioni.
\end{description}

Riguardo la stesura del testo, sono state adottate le seguenti convenzioni tipografiche:
\begin{itemize}
	\item i termini ambigui o di uso non comune menzionati vengono definiti nel glossario, situato alla fine del presente documento;
	\item per la prima occorrenza dei termini riportati nel glossario viene utilizzata la seguente nomenclatura: \emph{parola}\glsfirstoccur;
	\item i termini in lingua straniera o facenti parte del gergo tecnico sono evidenziati con il carattere \emph{corsivo}.
\end{itemize}