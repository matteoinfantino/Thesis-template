% !TEX encoding = UTF-8
% !TEX TS-program = pdflatex
% !TEX root = ../tesi.tex

%**************************************************************
\chapter{Analisi dei requisiti}
\label{cap:analisi-requisiti}
%**************************************************************

\intro{In questo capitolo vengono riportati i requisiti richiesti dal prodotto finale con l'utilizzo di una tabella per il tracciamento degli stessi.}\\

\section{Casi d'uso}
All'inizio dello stage sono stati discussi i casi d'uso del software con il tutor aziendale e gli altri stagisti partecipanti al progetto.
L'azienda intende sviluppare un'applicazione mobile per gestire il tracciamento dei contatti e una web application per permettere al personale sanitario di segnalare una persona risultata positiva a un tampone, sempre dietro conferma dell'infetto. Il mio ruolo nel progetto è stato integrare uno smart contract con le due parti. Tuttavia la \textit{blockchain} non è visibile da chi utilizza l'applicazione e non richiede un'interazione con l'utente. I casi d'uso sono stati stilati dai tirocinanti che hanno sviluppato l'applicazione e io mi sono limitato a rispettarli e a trovare i requisiti necessari allo \textit{smart contract} e all'integrazione con il software.
\section{Tracciamento dei requisiti}

Da un'attenta analisi dei requisiti effettuata sul progetto è stata stilata la tabella che traccia i requisiti. Sono stati individuati diversi tipi di requisiti e per distinguerli si è fatto utilizzo di un codice identificativo.\\
Il codice dei requisiti è così strutturato R(F/Q/V)(N/D/O) dove:
\begin{enumerate}
	\item[R =] requisito
    \item[F =] funzionale
    \item[Q =] qualitativo
    \item[V =] di vincolo
    \item[N =] obbligatorio (necessario)
    \item[D =] desiderabile
    \item[Z =] opzionale
\end{enumerate}
Nelle tabelle \ref{tab:requisiti-funzionali}, \ref{tab:requisiti-qualitativi} e \ref{tab:requisiti-vincolo} sono riassunti i requisiti delineati in fase di analisi.

\begin{center}
		\begin{longtable}{| c | p{30em} |}
			\caption{Tabella del tracciamento dei requisiti funzionali}
			\label{tab:requisiti-funzionali}\\
			\hline
			\textbf{Requisito} & \centering\textbf{Descrizione}\\
			\endfirsthead
			\hline
			\textbf{Requisito} & \centering\textbf{Descrizione}\\
			\endhead
			\endfoot
			
			\hline
			RFN-1     & L'address del proprietario del contract deve essere salvato\\
			\hline
			RFN-2     & Una persona deve essere inserita in catena con un id univoco \\
			\hline
			RFN-3     & Un inserimento deve essere rifiutato se è l'id della persona è già presente in catena \\
			\hline
			RFN-4     & Una persona infetta deve essere segnalata in blockchain \\
			\hline
			RFN-5     & Una segnalazione di infezione deve essere rifiutata se non è stata invocata da un medico autorizzato \\
			\hline
			RFN-6     & Una segnalazione di infezione deve essere rifiutata se viene inserito un id non presente in blockchain \\
			\hline
			RFN-7     & Quando viene rilevato un contatto tra due persone deve essere inserito in blockchain\\
			\hline
			RFN-8     & L'inserimento di un contatto viene rifiutato se viene effettuato da un indirizzo non corrispondente a quello della persona il cui contatto viene inserito \\
			\hline
			RFN-9     & L'inserimento di un contatto viene rifiutato se gli id non sono presenti in blockchain \\
			\hline
			RFN-10   & Deve essere possibile ottenere i contatti di una persona negli ultimi 14 giorni \\
			\hline
			RFN-11   & I contatti di una persona non devono essere forniti se vengono richiesti da qualcun altro\\
			\hline
			RFN-12   & Deve essere posssibile stabilire se una persona ha avuto contatti ritenuti a rischio \\
			\hline
			RFN-13   & Un rischio di contagio deve essere confermato anche dai contatti della persona infetta \\
			\hline
			RFN-14   & L'informazione sulla presenza di contatti a rischio deve essere effettuata solo dal proprietario dell'informazione \\
			\hline
			RFN-15   & Deve essere possibile calcolare la somma degli \emph{\gls{indici di contatto}}\glsfirstoccur di una persona negli ultimi gionri\\
			\hline
			RFN-16   & La somma degli indici di contatto deve essere effettuata solo dal proprietario dell'informazione \\
			\hline
			RFN-17   & Deve essere effettuato il deployment dello smart contract \\ 
			\hline
			RFN-18   & Lo smart contract deve essere utilizzato dall'applicazione mobile per l'inserimento dei contatti\\ 
			\hline
			RFN-19   & Lo smart contract deve essere utilizzato dalla web application per l'inserimento degli infetti \\ 
 			\hline
					

		\end{longtable}
	\end{center}

\begin{center}
	\begin{longtable}{| c | p{30em} |}
		\caption{Tabella del tracciamento dei requisiti qualitativi}
		\label{tab:requisiti-qualitativi}\\
		\hline
		\textbf{Requisito} & \centering\textbf{Descrizione}\\
		\endfirsthead
		\hline
		\textbf{Requisito} & \centering\textbf{Descrizione}\\
		\endhead
		\endfoot
		
		\hline
		RQN-1    & Lo smart contract deve essere il più possibile ottimizzato per limitare i consumi di gas \\
		\hline
		RQN-2    & Il codice deve essere versionato e reso disponibile nel \emph{\gls{repository}}\glsfirstoccur aziendale \\
		\hline
		RQN-3    & Deve essere fornito un documento tecnico \\
		\hline
	

	\end{longtable}
\end{center}

\begin{center}
	\begin{longtable}{| c | p{30em} |}
		\caption{Tabella del tracciamento dei requisiti di vincolo}
		\label{tab:requisiti-vincolo}\\
		\hline
		\textbf{Requisito} & \centering\textbf{Descrizione}\\
		\endfirsthead
		\hline
		\textbf{Requisito} & \centering\textbf{Descrizione}\\
		\endhead
		\endfoot
		
		\hline
		RVN-1    & Utilizzo del linguaggio Solidity e piattaforma Ethereum per lo sviluppo dello smart contract  \\
		\hline
		RVD-2    & Installazione dello smart contract in ambiente mobille (Android) \\
		\hline
		RVD-3    & Installazione dello smart contract in ambiente web \\
		\hline			
	\end{longtable}
\end{center}
