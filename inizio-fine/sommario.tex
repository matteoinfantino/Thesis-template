% !TEX encoding = UTF-8
% !TEX TS-program = pdflatex
% !TEX root = ../tesi.tex

%**************************************************************
% Sommario
%**************************************************************
\cleardoublepage
\phantomsection
\pdfbookmark{Sommario}{Sommario}
\begingroup
\let\clearpage\relax
\let\cleardoublepage\relax
\let\cleardoublepage\relax

\chapter*{Sommario}

La pandemia globale che ci ha colpito quest'anno ha avuto conseguenze in numerosi campi, oltre che nella vita di tutti noi. Il settore informatico ha cercato una soluzione che potesse favorire la ripresa della normale quotidianità: sembrava necessario trovare un modo per tracciare i contagi, riuscire a risalire alle persone entrate in contatto con malati di Covid-19 e interrompere la catena del contagio, tutto questo rivolgendo particolare attenzione al rispetto della privacy.  
La risposta è stata trovata nell'utilizzo di un'applicazione mobile che registri i contatti tra dispositivi con la tecnologia bluetooth LE, come nel caso della famosa applicazione Immuni attualmente utilizzata, seppur con poco successo, in Italia. Se adottate da una buona parte della popolazione, le applicazioni di contact tracing forniscono un supporto importante per il contenimento del contagio. Avere a disposizione i contatti di una persona infetta, significa agire tempestivamente segnalando ai soggetti a rischio il pericolo, al fine di limitare ulteriori contagi a catena.\\
L'azienda Sync Lab, dove ho svolto lo stage, ha incaricato un gruppo di tirocinanti di sviluppare un'applicazione di contact tracing chiamata SyncTrace. Come per Immuni, SyncTrace sfrutta la tecnologia bluetooth LE, ma fornisce funzionalità in più per il controllo dei contagi. Inoltre, l'azienda intende integrare nell'applicazione la tecnologia blockchain, sviluppando uno smart contract da utilizzare per il tracciamento dei contatti. Il mio ruolo all'interno del progetto è stato proprio lo studio e l'implementazione di uno smart contract per l'applicazione SyncTrace.
Il presente documento descrive il lavoro da me svolto durante il periodo di stage della durata di trecento ore.

%\vfill
%
%\selectlanguage{english}
%\pdfbookmark{Abstract}{Abstract}
%\chapter*{Abstract}
%
%\selectlanguage{italian}

\endgroup			

\vfill

